\chapter{Аналитическая часть}


\section{Формализация задачи поиска сообществ в графах}

\subsection{Социальный граф}

Строгого определения социального графа нет, однако он должен обладать следующими свойствами:

\begin{enumerate}
	\item среднее количество рёбер для перехода из одной вершины в другую не велико. Иллюстрацией этого свойства может послужить граф вида <<small-world>>;
	\item для многих вершин верно следующее: если А соединена с Б и А соединена с вершиной С, то с высокой вероятностью так же будут соединены Б и С;
	\item социальные графы поделены на сообщества.
\end{enumerate}

\subsection{Сообщества}

Сообщество на социальном графе -- это подмножество вершин графа количество связей между которыми плотнее, чем связи с остальной частью графа. 

Примером сообщества на графе может послужить группа друзей в социальной сети.

\section{Модулярность}

Модулярность -- это численная характеристика отображающая силу разделения графа на модули. Выражается эта характеристика по формуле~(\ref{eq:modularity}):

\begin{equation}
	\label{eq:modularity}
	Q = \frac{1}{2m}\sum_{i,j}(A_{ij} - \frac{d_id_j}{2m})\delta(C_i, C_j)
\end{equation}


\section{Методы поиска сообществ на графах}

\subsection{Label Propagation}



\section{Формализация данных}

В рамках курсовой работы были введены следующие сущности:

\begin{itemize}
	\item человек -- пользователь социальной сети;
	\item сообщество -- множество пользователей;
\end{itemize}

