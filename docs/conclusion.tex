\ssr{ЗАКЛЮЧЕНИЕ}

Цель работы была достигнута: была разработана база данных для анализа сообществ на графовой модели социальной сети. Для достижения данной цели были выполнены следующие задачи:

\begin{enumerate}
\item проанализирована предметная область;
\item проанализированы алгоритмы поиска сообществ на графах;
\item выбрана наиболее подходящая СУБД для решения поставленной задачи;
\item спроектирована база данных, описаны её основные сущности;
\item выбраны наиболее подходящие инструменты для разработки ПО;
\item исследована работа алгоритма, на разных входных данных.
\end{enumerate}

В результате исследования было выяснено, что число обнаруживаемых сообществ существенно зависит от плотности связей в графе: увеличение как межсообщностной, так и внутрисообщностной плотности приводит к укрупнению сообществ и снижению их количества, особенно на более высоких уровнях иерархии.