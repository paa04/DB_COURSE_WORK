\addcontentsline{toc}{chapter}{СПИСОК ИСПОЛЬЗОВАННЫХ ИСТОЧНИКОВ}
\begin{thebibliography}{}
\bibitem{alg} Методы выделения сообществ в
социальных графах. Никишин Евгений Сергеевич. \url{http://www.machinelearning.ru/wiki/images/8/8a/Nikishin_coursework_community_detection.pdf}

\bibitem{louv} On the Power of Louvain in the Stochastic Block
Model [Электронный ресурс]. Режим доступа: \url{https://groups.csail.mit.edu/tds/papers/Mallmann-Trenn/NEURIPS2020.pdf} (Дата обращения 2.06.25)

\bibitem{neo4j} Документация Neo4j [Электронный ресурс]. Режим доступа: \url{https://neo4j.com/docs/} (Дата обращения 2.06.25)

\bibitem{Csharp} Документация по C\# [Электронный ресурс]. Режим доступа: \url{https://learn.microsoft.com/ru-ru/dotnet/csharp/tour-of-csharp/} (Дата обращения 2.06.25)

\bibitem{cpu} Документация к процессору [Электронный ресурс]. Режим доступа: \url{https://www.intel.com/content/www/us/en/products/sku/80815/intel-core-i54590-processor-6m-cache-up-to-3-70-ghz/specifications.html} (Дата обращения 2.06.25)
	\bibitem{ubuntu} Сайт дистрибутива [Электронный ресурс]. Режим доступа: \url{https://ubuntu.com/} (Дата обращения 2.06.25)
	
\end{thebibliography}