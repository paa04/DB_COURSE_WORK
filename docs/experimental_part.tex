\chapter{Исследовательская часть}

\section{Технические характеристики}

\begin{itemize}
	\item процессор -- Intel\textregistered~Core\texttrademark~i5-4590 × 4~\cite{cpu};
	\item объём оперативной памяти -- 24 ГБ;
	\item операционная система -- Ubuntu 24.04.1 LTS~\cite{ubuntu}.	
\end{itemize}

\section{Исследование}

Было проведено исследование зависимости результата работы алгоритма от разряженности графа поданного на вход. В таблицах~\ref{tab:level1}~--~\ref{tab:level4plus} представлены результаты для различных слоёв иерархии сообществ.

\begin{itemize}
\item \textbf{Intra} (внутрисообщностная плотность) --- вероятность наличия ребра между двумя вершинами внутри одного сообщества. Чем выше значение Intra, тем более плотными являются сообщества, то есть больше связей внутри них.
\item \textbf{Inter} (межсообщностная плотность) --- вероятность наличия ребра между вершинами из разных сообществ. Чем выше значение Inter, тем сильнее сообщества связаны между собой, что затрудняет их выявление.
\end{itemize}

\begin{table}[ht]
\centering
\caption{Количество сообществ на уровне 1}
\label{tab:level1}
\begin{tabular}{|c|c|c|c|c|c|}
\hline
\textbf{} & \multicolumn{5}{c|}{\textbf{Inter}} \\
\hline
\textbf{Intra} & 0.005 & 0.01 & 0.05 & 0.1 & 0.15 \\
\hline
0.2 & 229 & 183 & 103 & 40 & 21 \\
0.3 & 151 & 135 & 96  & 39 & 18 \\
0.4 & 84  & 72  & 40  & 25 & 15 \\
0.5 & 51  & 43  & 30  & 17 & 13 \\
0.6 & 31  & 27  & 20  & 12 & 8  \\
\hline
\end{tabular}
\end{table}

\begin{table}[ht]
\centering
\caption{Количество сообществ на уровне 2}
\label{tab:level2}
\begin{tabular}{|c|c|c|c|c|c|}
\hline
\textbf{} & \multicolumn{5}{c|}{\textbf{Inter}} \\
\hline
\textbf{Intra} & 0.005 & 0.01 & 0.05 & 0.1 & 0.15 \\
\hline
0.2 & 30 & 22 & 9  & 5 & 1 \\
0.3 & 21 & 16 & 8  & 4 & 1 \\
0.4 & 10 & 8  & 5  & 2 & 1 \\
0.5 & 6  & 5  & 3  & 2 & 1 \\
0.6 & 5  & 3  & 2  & 1 & 1 \\
\hline
\end{tabular}
\end{table}

\begin{table}[ht]
\centering
\caption{Количество сообществ на уровне 3}
\label{tab:level3}
\begin{tabular}{|c|c|c|c|c|c|}
\hline
\textbf{} & \multicolumn{5}{c|}{\textbf{Inter}} \\
\hline
\textbf{Intra} & 0.005 & 0.01 & 0.05 & 0.1 & 0.15 \\
\hline
0.2 & 19 & 14 & 8 & 2 & 1 \\
0.3 & 11 & 9  & 7 & 2 & 1 \\
0.4 & 6  & 5  & 4 & 1 & 1 \\
0.5 & 4  & 3  & 2 & 1 & 1 \\
0.6 & 2  & 2  & 1 & 1 & 1 \\
\hline
\end{tabular}
\end{table}

\begin{table}[H]
\centering
\caption{Количество сообществ на уровне 4 и выше (усреднённо)}
\label{tab:level4plus}
\begin{tabular}{|c|c|c|c|c|c|}
\hline
\textbf{} & \multicolumn{5}{c|}{\textbf{Inter}} \\
\hline
\textbf{Intra} & 0.005 & 0.01 & 0.05 & 0.1 & 0.15 \\
\hline
0.2 & 11 & 9 & 8 & 2 & 1 \\
0.3 & 8  & 7 & 5 & 2 & 1 \\
0.4 & 5  & 4 & 3 & 1 & 1 \\
0.5 & 3  & 2 & 2 & 1 & 1 \\
0.6 & 2  & 2 & 1 & 1 & 1 \\
\hline
\end{tabular}
\end{table}

\section{Вывод}

По данным, полученным в ходе исследования, можно сделать следующие выводы:
\begin{itemize}
\item при увеличении значения параметра \textbf{Inter} (межсообщностной плотности) количество обнаруженных сообществ на всех уровнях иерархии значительно уменьшается. Это указывает на снижение чёткости границ между сообществами при росте количества межсообщностных связей;
\item при увеличении значения \textbf{Intra} (внутрисообщностной плотности) также наблюдается уменьшение количества сообществ, особенно на верхних уровнях иерархии. Это может быть связано с тем, что более плотные внутренние связи способствуют укрупнению сообществ;
\item на более глубоких уровнях иерархии количество сообществ резко сокращается, особенно при высоких значениях параметров Inter и Intra. Это свидетельствует о том, что структура сообщества становится менее детализированной.
\end{itemize}
